\documentclass[11pt]{article}

\include{macros}
\newcommand{\bitsoracle}{{\mathsf{Bits}}}
\newcommand{\indcpa}{{\mathrm{IND}\mbox{-}\mathrm{CPA}}}
\newcommand{\indrcpa}{{\mathrm{IND}\$\mbox{-}\mathrm{CPA}}}
\newcommand{\rorcpa}{\mathrm{RoR}}
\newcommand{\Func}{\mathsf{Func}}
 
%\newcommand{\Adv}{\mathrm{Adv}}
\newcommand{\ExpRoR}[2]{\mathsf{Exp}^{\mathrm{ror}}_{#1}{(#2)}}
\newcommand{\ExpPRF}[3]{\mathsf{Exp}^{\mathrm{prf}\mbox{-}#3}_{#1}{(#2)}}
\newcommand{\AdvPRF}[2]{\Adv^{\mathrm{prf}}_{#1}(#2)}
\newcommand{\AdvRoR}[2]{\Adv^{\mathrm{ror}}_{#1}(#2)}
\newcommand{\ExpVP}[2]{\mathsf{Exp}^{\mathrm{vp}}_{#1}{(#2)}}
\newcommand{\AdvVP}[2]{\Adv^{\mathrm{vp}}_{#1}(#2)}


%%%%%%%%%%%%%%%%%%%%%%%%%%%%%%%%%%%%%%%%%%%%%%%%%%%%%%%%
\title{\bf Problem Set 1 \\[2ex] 
       \rm\normalsize Modern Cryptography, Spring 2013}
\date{}
\author{}
\begin{document}
\maketitle
%%%%%%%%%%%%%%%%%%%%%%%%%%%%%%%%%%%%%%%%%%%%%%%%%%%%%%%%
\vspace*{-1in}
\begin{center}{Assigned: April 12, Due: April 24\\}
\end{center}

\section*{Problem 1 [20 points]} 
Consider the following security notion for a function family $E \colon
\calK \times \calD \to \calR$.  Let~$A$ be an adversary that takes one
oracle, namely $E_K(\cdot)$ for a random (and secret) key~$K \in
\calK$.  The adversary can adaptively query this oracle with various $X \in
\calD$, receiving $E_K(X)$ in response.  After asking its queries,
adversary~$A$ outputs a pair $(X^*,Y^*)$, and
``wins'' if $Y^* = E_K(X^*)$.  To prevent trivial wins for the
adversary, $X^*$ must not have been asked to the oracle.
Thus we define the following ``valid-pair'' experiment:

\begin{figure*}[h]
\center
\fpage{.6}{
\hpagess{.55}{.35}
{
\experimentv{$\ExpVP{E}{A}$:}\\
$Q \gets \emptyset$\\
$K \getsr \calK$\\
$(X^*,Y^*) \getsr A^{g(\cdot)}$\\
if $X^* \in \calQ$ then return 0\\
if $E_K(X^*) = Y^*$ then return 1\\
return 0
}
{
\medskip
\oraclev{$g(X)$}\\
$Y \gets E_K(X)$\\
$\calQ \gets \calQ \cup \{X\}$\\
return~$Y$
}
}
\end{figure*}
with the corresponding advantage measure
$\Adv_E^{\mathrm{vp}}(A) = \Pr\left[ \ExpVP{E}{A}=1 \right]$.
The resources counted are the number of queries~$q$, time
complexity~$t$, and the total length of the queries in bits, $\mu$.

You are to prove, via a security reduction, the informal statement
``security in the PRF sense implies security in the VP sense.''
Here is the shell of a theorem statement:

\begin{theorem}
Let $\calK$ be a non-empty set, and let $\calD,\calR \subseteq
\bits^*$. Let $E \colon \calK \times \calD \to
\calR$ be a function family.  Let~$A$ be a valid-pair adversary
for~$E$, running in time~$t$, asking~$q$ queries, these totaling
$\mu$ bits in length.  Then there exists a prf adversary~$B$ such
that
\[
\AdvVP{E}{A} \mbox{ ??? } \AdvPRF{E}{B} + \mbox{???}
\]
where~$B$ runs in time $t'= \mbox{???} $, asks $q'= \mbox{???} $ queries, these
totaling~$\mu'= \mbox{???} $ bits.
\end{theorem}

Complete the theorem statement, and prove it.

\section*{Problem 2}
Find a key~$K$ such that $DES_K(\cdot)=DES^{-1}_K(\cdot)$.  Explain
why your key satisifies this property.

\noindent
Note: there is a typo in Figure 2.1 of the the [BR] notes that is
important for this problem.  Just before the return statement, the
line should read: $C \gets \mathrm{IP}^{-1}(R_{16} \concat L_{16})$.
(That is, you ``undo'' the Feistel swap in the last round.)


\section*{Problem 3}
Bellare-Rogaway notes, problem 3.1.  Carefully describe the attack,
and give the PRP advantage of it.

\section*{Problem 4}
Bellare-Rogaway notes, problem 3.2.

\section*{Problem 5}
Bellare-Rogaway notes, problem 3.6.  This requires you to do another
security reduction.  No hand-waving, please!  Write up your solution
as a nice theorem, like the one above, or the one we saw in class.

Before you start trying to do the reduction, be clear on what kind of
adversary you assume is given, and what kind of adversary you want to build.  It's
easy to get confused, and end up with a security implication that
points in the wrong ``direction''.

\end{document}

