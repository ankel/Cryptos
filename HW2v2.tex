%Source for HW2, Winter 2012

\documentclass[11pt]{article}

\include{macros}
\newcommand{\bitsoracle}{{\mathsf{Bits}}}
\newcommand{\indcpa}{{\mathrm{IND}\mbox{-}\mathrm{CPA}}}
\newcommand{\indrcpa}{{\mathrm{IND}\$\mbox{-}\mathrm{CPA}}}
\newcommand{\rorcpa}{\mathrm{RoR}}
\newcommand{\Func}{\mathsf{Func}}

%\newcommand{\Adv}{\mathrm{Adv}}
\newcommand{\ExpRoR}[2]{\mathsf{Exp}^{\mathrm{ror}}_{#1}{(#2)}}
\newcommand{\ExpPRF}[3]{\mathsf{Exp}^{\mathrm{prf}\mbox{-}#3}_{#1}{(#2)}}
\newcommand{\ExpINDR}[2]{\mathsf{Exp}^{\mathrm{ind\$}\mbox{-}\mathrm{cpa}}_{#1}{(#2)}}
\newcommand{\ExpAuth}[2]{\mathsf{Exp}^{\mathrm{auth}}_{#1}{(#2)}}
\newcommand{\AdvPRF}[2]{\Adv^{\mathrm{prf}}_{#1}(#2)}
\newcommand{\AdvRoR}[2]{\Adv^{\mathrm{ror}}_{#1}(#2)}
\newcommand{\AdvINDR}[2]{\Adv^{\mathrm{ind\$}\mbox{-}\mathrm{cpa}}_{#1}(#2)}
\newcommand{\AdvAuth}[2]{\Adv^{\mathrm{auth}}_{#1}(#2)}

%%%%%%%%%%%%%%%%%%%%%%%%%%%%%%%%%%%%%%%%%%%%%%%%%%%%%%%%%%%%%%%%%%%%%%%%%%%
\title{\bf Problem Set 2, version 2 \\[2ex] 
       \rm\normalsize Modern Cryptography, Spring 2013}
\date{}
\author{}
\begin{document}
\maketitle


%%%%%%%%%%%%%%%%%%%%%%%%%%%%%%%%%%%%%%%%%%%%%%%%%%%%%%%%
\vspace*{-1in}
\begin{center}{Assigned: April 26, Updated: April 29, Due: May 6 \\}
\end{center}

\section*{Problem 1}
[BR] chapter 3, problem 8, part (a).

\section*{Problem 2} 
Let $\Pi=\left( \calK, \calE, \calD \right)$ be a symmetric encryption
scheme. Consider the following notion of security that we will call
\textit{indistinguishability of real or random encryptions} (RoR).  
Let~$A$ be an adversary that takes one oracle, has time complexity~$t$, asks~$q$
queries, these totalling~$\mu$ bits in length.  

\begin{figure*}[h]
\center
\fpage{.5}{
\hpagess{.4}{.55}
{
\experimentv{$\ExpRoR{\Pi}{A}$:}\\
$b \getsr \bits$\\
$K \getsr \calK$\\
$b' \getsr A^{\calO(\cdot)}$\\
if $b=b'$ return 1\\
return 0
}
{
\oraclev{$\calO(X)$}\\
If $b=1$ then $Y \getsr \calE_K(X)$\\
else\\
\hspace*{1ex}$Z \getsr \bits^{|X|}$\\
\hspace*{1ex}$Y \getsr \calE_K(Z)$\\
return~$Y$
}
}
\end{figure*}


Then define the
$\rorcpa$ advantage of~$A$ against~$\Pi$ by
\[
\AdvRoR{\Pi}{A} = 2\cdot\Pr\left[\ExpRoR{\Pi}{A}=1 \right] - 1
\]

Prove that security in the IND-CPA sense
implies RoR security.  This requires you to do a
reduction.  Please give a nice theorem statement, followed by your proof, 
and please be precise in your notation and your arguments.

(If you are feeling ambitious, you can try to prove the the converse:
that RoR security implies IND-CPA security.
Again, this is to be done by a reduction, but the argument is
trickier.  But skip this until you've finished the other problems.)

%%%%%%%%%%%%%%%%%%%%%%%%%%%%%%%%%%%%%%%%%%%%%%%%%%%%%%%%
\section*{Problem 3}
Let $\Pi=\left( \calK, \calE, \calD \right)$ be a symmetric encryption
scheme. Consider the following notion of security that we will call
\textit{indistinguishability from random bits} under a
chosen-plaintext attack (IND\$-CPA).  
Let~$A$ be an adversary that takes one oracle, has time complexity~$t$, asks~$q$
queries, these totalling~$\mu$ bits in length.  
\begin{figure*}[h]
\center
\fpage{.5}{
\hpagess{.4}{.55}
{
\experimentv{$\ExpINDR{\Pi}{A}$:}\\
$b \getsr \bits$\\
$K \getsr \calK$\\
$b' \getsr A^{\calO(\cdot)}$\\
if $b=b'$ return 1\\
return 0
}
{
\oraclev{$g(X)$}\\
$Y \getsr \calE_K(X)$\\
If $b=1$ Return $Y$\\
$Z \getsr \bits^{|Y|}$\\
return~$Z$
}
}
\end{figure*}


Then define the
IND\$-CPA advantage of~$A$ against~$\Pi$ by
\[
\AdvINDR{\Pi}{A} = 2\cdot\Pr\left[\ExpINDR{\Pi}{A}=1 \right] - 1
\]


Show that the IND\$-CPA security implies the IND-CPA security.

\if{0}
\begin{enumerate}
\item Show that the IND\$-CPA security notion, given in class, implies the
IND-CPA notion.  \\

\item Conversely, show that IND-CPA does \textit{not} imply
IND\$-CPA.  One way to do this is as follows: assume you have an
encryption scheme~$\Pi=(\calK,\calE,\calD)$ that is IND-CPA secure, and
then modify to produce a new scheme $\Pi' = (\calK',\calE',\calD')$
such that (1) scheme $\Pi'$ is provably IND-CPA secure, if $\Pi$ is;
and (2) scheme $\Pi'$ is provably \textit{not} IND\$-CPA secure.
\end{enumerate}
\fi

%%%%%%%%%%%%%%%%%%%%%%%%%%%%%%%%%%%%%%%%%%%%%%%%%%%%%%%%
\section*{Problem 4} 
Let $\Pi=(\calK,\calE,\calD)$ be a symmetric encryption scheme
based on an underlying function family $E \colon \bits^k \times
\bits^n \to \bits^n$.  On input $M \in (\bits^n)^+$ (ie, a string whose length
is a positive multiple of~$n$ bits), $\calE_K(M)$ operates as follows.  It parses~$M$ into $n$-bit blocks $M_1,\ldots,M_\ell$, samples a random IV $C_0 \getsr \bits^n$, and then for all $i \in \{1,2,\ldots,\ell\}$ it sets $C_i \gets C_{i-1} \xor E_K(M_i)$. Finally, it returns $C_0C_1\cdots C_\ell$ as the ciphertext.  Decryption occurs in the obvious way.  You are to prove or disprove this claim: if $E$ is a secure PRF, then $\Pi$ is IND-CPA secure.  To disprove it, give a carefully stated, nicely formatted attack.  To prove it, follow the steps used in the CBC proof and give a nice theorem statement.

\section*{Problem 5}
Let $\Pi=\left( \calK, \calE, \calD \right)$ be a symmetric encryption
scheme. Consider the following notion of security that we will call
\textit{authenticity of ciphertexts} (Auth).  

\begin{figure*}[h]
\center
\fpage{.5}{
\hpagess{.4}{.55}
{
\experimentv{$\ExpAuth{\Pi}{A}$:}\\
$b \getsr \bits$\\
$K \getsr \calK$\\
$b' \getsr A^{\calE_K(\cdot),\calO(\cdot)}$\\
if $b=b'$ return 1\\
return 0
}
{
\oraclev{$\calO(C)$}\\
If $b=1$ then $M \gets \calD_K(C)$\\
else $M \gets \bot$ \\
return~$M$
}
}
\end{figure*}
Then define the
Auth-advantage of~$A$ against~$\Pi$ by
\[
\AdvAuth{\Pi}{A} = 2\cdot\Pr\left[\ExpAuth{\Pi}{A}=1 \right] - 1 \,.
\]
To prevent trivial wins, we forbid Auth-adversaries from asking~$C$ to
the right oracle if~$C$ was previously returned by the left oracle.
(In other words, replay doesn't count.)

Intuitively, this notion captures the class of attacks in which an
attacker tries to ``forge'' a ciphertext, and have that forgery be
accepted by the receiver.  
Encryption schemes that are Auth-secure should withstand such attacks,
thereby providing an authenticity guarantee to the ciphertexts it creates.
The forgery might be produced by altering
or combining ciphertexts that it has previously observed, which is why we give an
oracle for $\calE_K$. 

A folklore suggestion for achieving Auth security is to take a strong
encryption scheme and introduce ``redundancy'' into the plaintext.
So, consider the following variant of random-IV CBC-mode.
On input a plaintext $M \in (\bits^n)^+$, where~$n$ is the blocksize
of the underlying blockcipher: (1) $M' \gets M \concat 0^n$, and (2)
output the ciphertext~$C$ resulting from normal CBC mode encryption of
$M'$. (Recall that, when $u,v$ are strings, $u \concat v$ is the
concatenation of these.)  Decryption of ciphertext~$C$ proceeds in the obvious way: 
(1) decrypt~$C$ using normal CBC mode decryption to get $M'$, 
(2) if $M'$ is of the form $X \concat 0^n$, return~$X$; else
return~$\bot$.

Does this CBC-with-redundancy encryption scheme achieve authenticity
of ciphertexts?  If yes, prove it.  If no, exhibit an efficient attack
and give it's Auth-advantage.
\end{document}
