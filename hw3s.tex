%Source for HW3, Winter 2012

\documentclass[11pt]{article}

\include{macros}
\newcommand{\bitsoracle}{{\mathsf{Bits}}}
\newcommand{\indcpa}{{\mathrm{IND}\mbox{-}\mathrm{CPA}}}
\newcommand{\indrcpa}{{\mathrm{IND}\$\mbox{-}\mathrm{CPA}}}
\newcommand{\rorcpa}{\mathrm{RoR}}
\newcommand{\Func}{\mathsf{Func}}

%\newcommand{\Adv}{\mathrm{Adv}}
\newcommand{\ExpUFCMA}[2]{\mathsf{Exp}^{\mathrm{ufcma}}_{#1}{(#2)}}
\newcommand{\ExpRoR}[2]{\mathsf{Exp}^{\mathrm{ror}}_{#1}{(#2)}}
\newcommand{\ExpPRF}[3]{\mathsf{Exp}^{\mathrm{prf}\mbox{-}#3}_{#1}{(#2)}}
\newcommand{\AdvPRF}[2]{\Adv^{\mathrm{prf}}_{#1}(#2)}
\newcommand{\AdvRoR}[2]{\Adv^{\mathrm{ror}}_{#1}(#2)}
\newcommand{\AdvUFCMA}[2]{\Adv^{\mathrm{ufcma}}_{#1}(#2)}

%%%%%%%%%%%%%%%%%%%%%%%%%%%%%%%%%%%%%%%%%%%%%%%%%%%%%%%%%%%%%%%%%%%%%%%%%%%

\title{\bf Problem Set 3 \\[2ex] 
       \rm\normalsize Modern Cryptography, Spring 2013}
\date{}
\author{}
\begin{document}
\maketitle


%%%%%%%%%%%%%%%%%%%%%%%%%%%%%%%%%%%%%%%%%%%%%%%%%%%%%%%%
\vspace*{-1in}

\begin{center}{Group: Binh Tran, David Baldwin\\}
\end{center}

\section*{Problem 1}
Let $E \colon \bits^n \times \bits^n \to \bits^n$ be a blockcipher,
and consider the following three functions, each mapping $\bits^n
\times \bits^n$ to $\bits^n$:

\begin{enumerate}
\item $f_1(V,M) = E_V(V \xor M) \xor M $\\

This function is not collision resistant. Consider the following attack that makes 1 query:\\
\begin{enumerate}
\item $V \gets 0^n$
\item $M_1 \getsr \bits^n$
\item $M_2 \getsr E^-1_V(M_1)$
\item return $(V, M_1)$ and $(V, M_2)$
\end{enumerate}

\textbf{Analysis}

\begin{eqnarray*}
f_1(V, M_1) = E_V(V \xor M_1) \xor M_1\\
=E_V(M_1) \xor M_1
\end{eqnarray*}

\item $f_2(V,M) = E_V(V \xor M) \xor V \xor M$

Consider $f'_2(V,N)=E_V(N) \xor N$ where $N=V \xor M$. $f'_2$ is the same as $f_2$, thus\\If $\exists (V, M), (V', M')$ such that $f_2(V,M)=f_2(V', M')$ then\\ $\exists (V, N), (V', N)$ such that $f'_2(V,N)=f'_2(V', N')$\\

In that case,\\
\begin{eqnarray*}
E_V(N) \xor N &=& E_{V'}(N') \xor N'\\
E_V(N)&=& E_{V'}(N') \xor N' \xor N\\
\end{eqnarray*}

If $E_V$ and $E_{V'}$ are both strong PRF, then 
\begin{eqnarray*}
Pr(N': E_V(N)= E_{V'}(N') \xor N' \xor N)&=&Pr(T: E_V(N)=T)\\
&=&\frac{1}{2^n}
\end{eqnarray*}
and the adversary will need to makes about $\sqrt{2^n}$ queries before the collision happens.
\item $f_3(V,M) = E_V(M) \xor V$
\\This function is also not collision resistant. Consider the following attack that makes 2 queries:\\
\begin{enumerate}
\item $V_1, V_2 \getsr \bits^n$ where $V_1 \neq V_2$\\
\item $M_1 \gets E^-1_{V_1}(V_1)$\\
\item $M_2 \gets E^-1_{V_2}(V_2)$\\
\end{enumerate}

\textbf{Analysis}
\begin{eqnarray*}
E_{V_1}(M_1) \xor V_1 &=& E_{V_1}(E^-1_{V_1}(V_1)) \xor V_1 \\
&=& V_1 \xor V_1 \\
&=& 0^n\\
\end{eqnarray*}
and
\begin{eqnarray*}
E_{V_2}(M_2) \xor V_2 &=& E_{V_2}(E^-1_{V_2}(V_2)) \xor V_2 \\
&=& V_2 \xor V_2 \\
&=& 0^n\\
\end{eqnarray*}
This causes the collision\\
\end{enumerate}

\newcommand{\oPi}{\overline{\Pi}}
\newcommand{\ocalK}{\overline{\mathcal{K}}}
\newcommand{\ocalE}{\overline{\mathcal{E}}}
\newcommand{\ocalD}{\overline{\mathcal{D}}}

\vspace*{-.25in}
\section*{Problem 2} Let~$F \colon \bits^k \times \bits^* \to \bits^n$
be a function family, and let $\Pi = (\calK,\calE,\calD)$ be an
encryption scheme.  Consider the following encryption scheme $\oPi =
(\ocalK,\ocalE,\ocalD)$ where 
\[
\ocalE_{K1,K2}(M) = \calE_{K1}(M)\concat F_{K2}(K1 \concat M)\, .
\]  
To decrypt, $\ocalD_{K1,K2}(C)$
operates as follows: (1) parse~$C$ into $Y \concat T$ where $|T|=n$; (2) $X
\gets \calD_{K1}(Y)$, and if~$X = \bot$ then return~$\bot$; (3) if
$F_{K2}(K1 \concat X)=T$ then return~$X$, else return $\bot$.

\paragraph{Part 1.} Assume that~$F$ is a good PRF, and that $\Pi$ is IND\$-CPA secure.
Are these assumptions sufficient to prove that $\oPi$ is IND\$-CPA
secure?

\newcommand{\INDR}{IND\$-CPA}
\newcommand{\ExpINDz}[2]{\mathsf{Exp}^{\mathrm{ind}\mbox{-}\mathrm{cpa0}}_{#1}{(#2)}}
\newcommand{\ExpINDo}[2]{\mathsf{Exp}^{\mathrm{ind}\mbox{-}\mathrm{cpa1}}_{#1}{(#2)}}
\newcommand{\ExpINDR}[2]{\mathsf{Exp}^{\mathrm{ind\$}\mbox{-}\mathrm{cpa}}_{#1}{(#2)}}
\newcommand{\AdvINDR}[2]{\Adv^{\mathrm{ind\$}\mbox{-}\mathrm{cpa}}_{#1}(#2)}
\newcommand{\Exp}[3]{\mathsf{Exp}^{\mathrm{#1}}_{#2}{(#3)}}
\newcommand{\Adva}[3]{\mathsf{Adv}^{\mathrm{#1}}_{#2}{(#3)}}

\textsc{Answer} 
\\$\oPi$ is not \INDR. Here's an adversary A for $\ExpINDR{\oPi}{(A)}$ that makes 2 query totaling $2n$ bits length, $O(1)$ calculation time:
\begin{enumerate}
\item M $\getsr \bits^n$
\item $X_l \concat X_r \gets X \gets \calO(M)$ where $|X_r| = n$
\item $Y_l \concat Y_r \gets Y \gets \calO(M)$ where $|Y_r| = n$
\item if $X_r = Y_r$ then \texttt{ return 1} else \texttt{ return 0}
\end{enumerate}

\textsc{Analysis}\\
\begin{enumerate}
\item In $\ExpINDo{\oPi}{A}$\\
$X_r = F_{K2}(K1 \concat M)$ and $Y_r = F_{K2}(K1 \concat M)$ by the construction of $\oPi$ thus\\
\begin{center}
 $Pr(\ExpINDo{\oPi}{A}=1) =1$
\end{center}

\item In $\ExpINDz{\oPi}{A}$\\
If $\calO (M)$ is random then A returns 1 when 2 random bit strings has the same suffix 

\begin{eqnarray*}
Pr(\ExpINDo{\oPi}{A}=1) &=& Pr (X_r = Y_r)\\
&=& Pr (X_r = T: Y_r = T)\\
&=& \frac{1}{2^n}
\end{eqnarray*}

\item $\AdvINDR{\oPi}{A}$\\

\begin{eqnarray*}
\AdvINDR{\oPi}{A} &=& Pr(\ExpINDo{\oPi}{A}=1)-Pr(\ExpINDz{\oPi}{A}=1)\\
&=& 1 - \frac{1}{2^n}
\end{eqnarray*}

\end{enumerate}

\paragraph{Part 2.} Are these assumptions sufficient to prove that $\oPi$ is AUTH-secure?
If so, please give a detailed proof sketch (as above), and what you
expect will be the security bound.  If not, please give an efficient attack.

\textsc{Answer}\\
$\oPi$ is AUTH-secure as shown using the following security reduction:

\textit{Statement: } If there exists an adversary A, making $q$ queries totaling $\mu$ bits length and doing calculation in $t$ time, in $\Exp{AUTH}{\oPi}{A}$ then there exists an adversary B, making $q$ queries totaling $/mu + qn$ bits length and doing calculation in $t + O(\frac{\mu}{q})$ time, in $\Exp{PRF}{F}{B}$ that satisfy

\begin{eqnarray*}
\Adva{PRF}{F}{B} = \Adva{AUTH}{\oPi}{A} -  \frac{1}{2^n}
\end{eqnarray*}

\textit{Building B:}\\
\begin{itemize}
\item $K_1 \gets \bits^n$
\item Run A
\item If A asks for encryption of M
\begin{itemize}
\item $X \gets \calE_{K_1}(M)$ where $\calE$ is simulated by B
\item $Y \gets \calO(K_1 \concat M)$
\item give $(X \concat Y)$ to A
\end{itemize}
\item If A asks for decryption of Z
\begin{itemize}
\item $C \concat D \gets Z$ where $|D|=n$
\item $I \gets \calE^{-1}_{K_1}(C)$ where $\calE^{-1}$ is simulated by B
\item $J \gets \calO(K_1 \concat I)$
\item if $D==J$ then gives $I$ to A else give $\bot$ to A
\end{itemize}
\item Run until A halts.
\item $C_0 \concat T_0 \gets A$ where $|T_0|=n$
\item $M_0 \gets \calE^{-1}_{K_1}(C_0)$ where $\calE^{-1}$ is simulated by B
\item $T_1 \gets \calO(M_0)$
\item if $T_0 == T_1$ returns 1 else return 0
\end{itemize}

\textit{Analysis}\\
\begin{itemize}
\item In $\Exp{PRF-1}{F}{B}$\\
Since B simulate $\oPi$ exactly, and B only returns 1 when A manages to break $\oPi$ in $Exp{AUTH}{\oPi}{A}$
\begin{equation}\label{prf1}
Pr(\Exp{PRF-1}{F}{B}=1)= \Adva{AUTH}{\oPi}{A}
\end{equation}
\item In $\Exp{PRF-0}{F}{B}$\\
In this case, B returns 1 when a tag chosen randomly happens to match the tag returned by A\\
\begin{equation}\label{prf0}
Pr(\Exp{PRF-0}{F}{B}=1) = \frac{1}{2^n}
\end{equation}
\item $\Adva{PRF}{F}{B}$\\
From \ref{prf1} and \ref{prf0}
\begin{equation}
\Adva{PRF}{F}{B} = \Adva{AUTH}{\oPi}{A} -  \frac{1}{2^n}
\end{equation}
\item Run time behavior:\\
\begin{itemize}
\item Every time A asks a query, either to the Encrypt or Decrypt oracle, B asks $\calO$ a query. Thus B makes $q$ queries in total.
\item Every time A asks a $m$-bit $M$ query, B makes a $(|K \concat M| = n + m)$-bit query. After $q$ queries, the total length is $\mu + qn$ bits
\item Every time A asks a $l$-block query, B has to spend an extra $O(l)$ time to encrypt / decrypt those blocks. In the end A queries total $\frac{\mu}{q}$ blocks thus B spends $t+O(\frac{\mu}{q})$ time
\end{itemize}
\end{itemize}

\paragraph{Part 3.}
$\ocalE'_{K1,K2}(M) = \calE_{K1}(M)\concat F_{K2}(K1 \concat |M|)$ is secure in neither IND\$-CPA nor AUTH\\

For IND\$-CPA, the same attack on $\oPi$ also works on $\oPi'$ The reason is that the tag in $\oPi'$ is generated by the length of the plain text, thus same plain text - having the same length - will result in same tag.\\

The following attacker can bypass $\oPi'$ in AUTH game - Assume that $|X|=|Y| \rightarrow |\calE(X)|=|\calE(Y)|$

\begin{itemize}
\item $M_1 \getsr \bits^n$
\item $C_1 \concat T_1 \gets \calE'(M_1)$ where $|T_1|=n$
\item $C_2 \getsr \bits^{n}$
\item return $C_2 \concat T_1$
\end{itemize}

The reason this works is that $T_1$ is the universal tag for all n-bit plain text, and by our assumption, n-bit cipher text. $C_2$ is one of such cipher text, and thus will be accepted by the decryption.

\end{document}
