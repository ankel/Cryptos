%Source for HW3, Winter 2012

\documentclass[11pt]{article}

% page size parameters
\usepackage{latexsym,setspace,xspace}
\setlength\itemindent{0.1in}
\setlength{\itemsep}{0in}

\setlength\topmargin{-0.75in}
\setlength\oddsidemargin{0.0in}
\setlength\evensidemargin{0.0in}
\setlength\parindent{0.0in} 
\setlength\textwidth{6.5in}
\setlength\textheight{9.25in}
\setlength{\parskip}{\medskipamount}

\newenvironment{newmath}{\begin{displaymath}%
\setlength{\abovedisplayskip}{4pt}%
\setlength{\belowdisplayskip}{4pt}%
\setlength{\abovedisplayshortskip}{3pt}%
\setlength{\belowdisplayshortskip}{3pt} }{\end{displaymath}}


% header hackery

\newlength{\toppush}
\setlength{\toppush}{2\headheight}
\addtolength{\toppush}{\headsep}


\def\subjnum{CS 4/585}
\def\subjname{Cryptography}

\def\doheading#1#2#3{\vfill\eject\vspace*{-\toppush}%
\vbox{\hbox to\textwidth{{\bf}\subjnum: \subjname \hfil Handout #1\strut}%
\hbox to\textwidth{PSU --- Tom Shrimpton\hfil#3\strut}%
    \hrule}}

\newcommand{\htitle}[1]{\vspace*{3.25ex}%
  \begin{center}{\Large\bf #1}\end{center}}


% page style that displays just footers
\makeatletter
\def\ps@justfooters{\let\@mkboth\@gobbletwo\def\@oddhead{}%
	\def\evenhead{}}
\makeatother

% \handout{handout label}{title}{date}

\newcommand{\handout}[3]{\thispagestyle{justfooters}%
  \markboth{\subjnum\ Handout \protect\ref{#1}: #2}%
    {\subjnum\ Handout \protect\ref{#1}: #2}%
  \pagestyle{myheadings}%
  \doheading{\protect\ref{#1}}{#2}{#3}%
  \htitle{#2}}

\newcommand{\procfont}[1]{\textbf{#1}}
\newcommand{\nfont}[1]{{\footnotesize #1}}


\newcommand{\dash}{{\mbox{-}}}
\newcommand{\taglen}{{\tau}}
\newcommand{\Pad}{P}
\newcommand{\Ctr}{{S}}

\newcommand{\outputs}{\Rightarrow}
\newcommand{\prf}{\mathrm{prf}}
\newcommand{\prp}{\mathrm{prp}}
\newcommand{\tprp}{\mathrm{tprp}}
\newcommand{\sprp} {{\pm\mathrm{prp}}}
\newcommand{\stprp}{{\pm\mathrm{tprp}}}
\newcommand{\ind} {{\mathrm{ind\mbox{-}{cpa}}}}
\newcommand{\auth} {{\mathrm{auth}}}
\newcommand{\indr}{{\mathrm{ind\$\mbox{-}{cpa}}}}
\newcommand{\indrcca}{{\mathrm{ind\$\mbox{-}{cca}}}}
\newcommand{\indcca}{{\mathrm{ind\mbox{-}{cca2}}}}
\newcommand{\nmcca}{{\mathrm{nm\mbox{-}cca}}}

\newcommand{\xxx}{{\mathrm{xxx}}}
\newcommand{\Encrypt}{\mathrm{Encrypt}}
\newcommand{\Decrypt}{\mathrm{Decrypt}}

\newcommand{\ERROR}{{\textsc{Error}}}
\newcommand{\invalid}{{\textsc{Invalid}}}

\newcommand{\CBC}{\mathrm{CBC}}
\newcommand{\CTR}{\mathrm{CTR}}
\newcommand{\CT}{{\mathit{CT}}}

\newcommand{\captionfont}{\small\sf }

\newcommand{\sep}{\:}
\newcommand{\kwfont}{\bf}
\newcommand{\AND}{\mbox{\kwfont and}}
\newcommand{\ALGORITHM}{\mbox{\kwfont Algorithm}}
\newcommand{\BEGIN}{\mbox{\kwfont begin}}
\newcommand{\DO}{\mbox{\kwfont do}}
\newcommand{\ELSE}{\mbox{\kwfont else}}
\newcommand{\END}{\mbox{\kwfont end}}
\newcommand{\FOR}{\mbox{\kwfont for}}
\newcommand{\IF}{\mbox{\kwfont if}}
\newcommand{\RETURN}{\mbox{\kwfont return}}
\newcommand{\THEN}{\mbox{\kwfont then}}
\newcommand{\TO}{\mbox{\kwfont to}}

\newcommand{\ZERO}{{\textbf{0}}}
\newcommand{\ONE}{{\textbf{1}}}
\newcommand{\error}{\mbox{\sc error}}
\newcommand{\true}{{\mathtt{true}}}
\newcommand{\false}{{\mathtt{false}}}
\newcommand{\undef}{{\textsc{undefined}}}
\newcommand{\dotdot}{{\,..\,}}
\newcommand{\xor}{{\,\oplus\,}}
\newcommand{\bits}{{\{0,1\}}}
\newcommand{\bitss}{\bits^*}
\newcommand{\e}{{\epsilon}}
\newcommand{\outputlen}{\varsigma}

\newlength{\saveparindent}
\setlength{\saveparindent}{\parindent}
\newlength{\saveparskip}
\setlength{\saveparskip}{\parskip}



% ====================================================================

% Theorem environments

\newtheorem{thm}{Theorem} %[section]
\newtheorem{lem}[thm]{Lemma}
\newtheorem{cor}[thm]{Corollary}
\newtheorem{propo}[thm]{Proposition}
\newtheorem{defn}[thm]{Definition}
\newtheorem{assm}[thm]{Assumption}
\newtheorem{clm}[thm]{Claim}
\newtheorem{rem}[thm]{Remark}

\newenvironment{theorem}{\begin{thm}\begin{sl}}%
{\end{sl}\end{thm}}
\newenvironment{lemma}{\begin{lem}\begin{sl}}%
{\end{sl}\end{lem}}
\newenvironment{corollary}{\begin{cor}\begin{sl}}%
{\end{sl}\end{cor}}
\newenvironment{proposition}{\begin{propo}\begin{sl}}%
{\end{sl}\end{propo}}
\newenvironment{definition}{\begin{defn}\begin{em}}%
{\end{em}\end{defn}}
\newenvironment{assumption}{\begin{assm}\begin{em}}%
{\end{em}\end{assm}}
\newenvironment{claim}{\begin{clm}\begin{sl}}%
{\end{sl}\end{clm}}
\newenvironment{remark}{\begin{rem}\begin{em}}%
{\end{em}\end{rem}}


\newcommand{\secref}[1]{\mbox{Section~\ref{#1}}}
\newcommand{\apref}[1]{\mbox{Appendix~\ref{#1}}}
\newcommand{\thref}[1]{\mbox{Theorem~\ref{#1}}}
\newcommand{\prref}[1]{\mbox{Proposition~\ref{#1}}}
\newcommand{\defref}[1]{\mbox{Definition~\ref{#1}}}
\newcommand{\corref}[1]{\mbox{Corollary~\ref{#1}}}
\newcommand{\lemref}[1]{\mbox{Lemma~\ref{#1}}}
\newcommand{\clref}[1]{\mbox{Claim~\ref{#1}}}
\newcommand{\propref}[1]{\mbox{Proposition~\ref{#1}}}
\newcommand{\figref}[1]{\mbox{Figure~\ref{#1}}}
\newcommand{\eqref}[1]{\mbox{(\ref{#1})}}

\newcommand{\maybespace}{{\ifnum\springer=0{{\ }}\fi}}


\def\sqed{{\hspace{5pt}\rule[-1pt]{3pt}{9pt}}}
% deluxe proof environment
\def\qsym{\vrule width0.6ex height1em depth0ex}
\newcount\proofqeded
\newcount\proofended
\def\qed{ {\hspace{5pt}\rule[-1pt]{3pt}{9pt}}
\end{rm}\addtolength{\parskip}{-0pt}
\setlength{\parindent}{\saveparindent}
\global\advance\proofqeded by 1 }

\newenvironment{proof}%
 {\proofstart}%
 {\ifnum\proofqeded=\proofended\qed\fi \global\advance\proofended by 1
  \medskip}
\makeatletter
\def\proofstart{\@ifnextchar[{\@oprf}{\@nprf}}
\def\@oprf[#1]{\begin{rm}\protect\vspace{6pt}\noindent{\bf Proof of #1:\
}%
\addtolength{\parskip}{5pt}\setlength{\parindent}{0pt}}
\def\@nprf{\begin{rm}\protect\vspace{6pt}\noindent{\bf Proof:\ }%
\addtolength{\parskip}{5pt}\setlength{\parindent}{0pt}}

% ========================================================================

% Lists

\newcounter{ctr}
\newcounter{savectr}
\newcounter{ectr}

\newenvironment{newenum}{%
\begin{list}{{\rm (\arabic{ctr})}\hfill}{\usecounter{ctr} \labelwidth=18pt%
\labelsep=7pt \leftmargin=25pt \topsep=3pt%
\setlength{\listparindent}{\saveparindent}%
\setlength{\parsep}{\saveparskip}%
\setlength{\itemsep}{2pt} }}{\end{list}}

\newenvironment{bignewenum}{%
\begin{list}{{\bf (\arabic{ctr})}\hfill}{\usecounter{ctr} \labelwidth=18pt%
\labelsep=7pt \leftmargin=25pt \topsep=6pt%
\setlength{\listparindent}{\saveparindent}%
\setlength{\parsep}{\saveparskip}%
\setlength{\itemsep}{6pt} }}{\end{list}}

\newenvironment{tiret}{%
\begin{list}{\hspace{2pt}\rule[0.5ex]{6pt}{1pt}\hfill}{\labelwidth=18pt%
\labelsep=7pt \leftmargin=25pt \topsep=3pt%
\setlength{\listparindent}{\saveparindent}%
\setlength{\parsep}{\saveparskip}%
\setlength{\itemsep}{1pt} }}{\end{list}}

\newenvironment{onelist}{%
\begin{list}{{\rm (\arabic{ctr})}\hfill}{\usecounter{ctr} \labelwidth=15pt%
\labelsep=2pt \leftmargin=17pt \topsep=1pt% 
\setlength{\listparindent}{\saveparindent}%
\setlength{\parsep}{\saveparskip}%
\setlength{\itemsep}{0pt} }}{\end{list}} 

\newenvironment{twolist}{%
\begin{list}{{\rm (\arabic{ctr}.\arabic{ectr})}%
\hfill}{\usecounter{ctr} \labelwidth=37pt%
\labelsep=7pt \leftmargin=44pt \topsep=0pt%
\setlength{\listparindent}{\saveparindent}%
\setlength{\parsep}{\saveparskip}%
\setlength{\itemsep}{0pt} }}{\end{list}}

\newenvironment{newitemize}{%
\begin{list}{$\bullet$}{\labelwidth=19pt%
\labelsep=7pt \leftmargin=26pt \topsep=3pt%
\setlength{\listparindent}{\saveparindent}%
\setlength{\parsep}{\saveparskip}%
\setlength{\itemsep}{3pt} }}{\end{list}}

% =========================================================================

% General

\newlength{\savejot}
\setlength{\savejot}{\jot}


\newcommand{\heading}[1]{{\vspace{10pt}\noindent{\textsc{#1}}}}
\newcommand{\noskipheading}[1]{{\noindent{\textsc{#1}}}}

\newcommand{\headingg}[1]{{\textsc{#1}}}
\newcommand{\Heading}[1]{{\vspace{8pt}\noindent\textbf{#1}}}

\newcommand{\emptystring}{\varepsilon}
\newcommand{\concat}{\:\|\:}
\newcommand{\Concat}{\;\|\;}
\newcommand{\Colon}{{\,:\;\,}}

\newcommand{\N}{{{\sf N}}}
\newcommand{\R}{{{\rm\bf R}}}
\newcommand{\Y}{{{\sf Y}}}
\newcommand{\Z}{{{Z}}}

\newcommand{\calC}{{\cal C}}
\newcommand{\calD}{{\cal D}}
\newcommand{\calE}{{\cal E}}
\newcommand{\calF}{{\cal F}}
\newcommand{\calI}{{\cal I}}
\newcommand{\calO}{{\cal O}}
\newcommand{\calP}{{\cal P}}
\newcommand{\calQ}{{\cal Q}}
\newcommand{\calR}{{\cal R}}
\newcommand{\calG}{{\cal G}}
\newcommand{\calK}{{\cal K}}
\newcommand{\calM}{{\cal M}}
\newcommand{\calN}{{\cal N}}
\newcommand{\calT}{{\cal T}}

\newcommand{\D}{{\calD}}
\newcommand{\E}{{\calE}}
\newcommand{\K}{{\calK}}
\newcommand{\I}{{\calI}}
\renewcommand{\R}{{\calR}}
\renewcommand{\O}{{\calO}}
\newcommand{\M}{{\mathsf{Message}}}
\newcommand{\C}{{\mathsf{Ciphertext}}}


\newcommand{\Key}{\mathsf{Key}}
\newcommand{\IVspace}{\mathsf{IV}}
\newcommand{\Nonce}{\mathsf{Nonce}}
\newcommand{\Plaintext}{\mathsf{Plaintext}}
\newcommand{\Ciphertext}{\mathsf{Ciphertext}}

\newcommand{\goesto}{{\rightarrow}}
\newcommand{\eqdef}{\stackrel{\rm def}{=}}
\newcommand{\getsr}{{\:\stackrel{{\scriptscriptstyle \hspace{0.2em}\$}} {\leftarrow}\:}}
\newcommand{\getsd}{{\:\stackrel{{\scriptscriptstyle \hspace{0.2em}D}} {\leftarrow}\:}}
\renewcommand{\choose}[2]{{{#1}\atopwithdelims(){#2}}}
\newcommand{\abs}[1]{{\displaystyle \left| {#1} \right| }}
\newcommand{\EE}[1]{{\E\left[{#1}\right]}}

\newcommand{\Damgard}{{\mbox{Damg{\aa}rd}}}

\newcommand{\strtonum}[1]{{ \mathsf{str2num}\left({#1}\right) }}
\newcommand{\numtostr}[2]{{ \mathsf{num2str}_{\scriptscriptstyle #2} 
                              \left( {#1} \right) }}

\newcommand{\Adv}{{\mathbf{\bf Adv}}}

\newcommand{\Randword}{\mathrm{Rand}}
\newcommand{\Permword}{\mathrm{Perm}}
\newcommand{\Randd}[2]{\ensuremath{\Randword({#1},{#2})}}
\newcommand{\Rand}[1]{\ensuremath{\Randword({#1})}}
\newcommand{\Rn}{{\calR_n}}
\newcommand{\Pn}{{\calP_n}}
\newcommand{\Perm}[1]{\ensuremath{\Permword({#1})}}
\newcommand{\Permm}[2]{\ensuremath{\Permword({#1},{#2})}}

\newcommand{\GF}{{\mathrm{GF}}}

\newcommand{\code}[1]{{\langle{#1}\rangle}}

\newcommand{\then}{{;\;}}
\newcommand{\andthen}{{\::\;\;}}

\newcommand{\ie}{i.e.}
\newcommand{\eg}{e.g.}
\newcommand{\adv}{\ensuremath{\mathrm{Adv}}}
\newcommand{\hdir}{\mbox{\vspace{6mm}\~{ }}}



% ========================================================================

\makeatletter
% from  /usr/local/lib/tex/inputs/article.sty  and
%       /usr/local/lib/tex/inputs/latex.tex

\newcommand{\imply}{\boldmath{\Rightarrow}}

\newcommand{\IV}{{V}}

\newcommand{\CBCexplicitIV} {{\mathrm{CBC }}}
\newcommand{\CBCrandomIV}   {{\mathrm{CBC \$   }}}
\newcommand{\CBCencipherIVone} {{\mathrm{CBC 1}}}
\newcommand{\CBCencipherIVtwo} {{\mathrm{CBC 2}}}
\newcommand{\CBCencipherIVtwox} {{\mathrm{CBC 2 X}}}

\newcommand{\simulate}[3]{{\mbox{\textsl{SimulateOracles}}(#1,#2,#3)}}
\renewcommand{\Y}{{\cal Y}}
\newcommand{\setinvalid}{\mathrm{Invalid}}
\newcommand{\X}{{\cal X}}
\newcommand{\range}{\mathrm{Range}}
\newcommand{\domain}{\mathrm{Domain}}
\newcommand{\ran}{\mathrm{Range}}
\newcommand{\dom}{\domain}
\newcommand{\gamef}{\mathrm{GameF}}
\newcommand{\gamer}{\mathrm{Game\$}}
\newcommand{\cran}{\overline{\range}}
\newcommand{\cdom}{\overline{\domain}}
\newcommand{\bad}{\mathit{bad}}

\newcommand{\PrK}{{\mbox{$\Pr_K$}}}
\newcommand{\PrF}{{\mbox{$\Pr_F$}}}
\newcommand{\Bad}{{\mathsf{BAD}}}

\newcommand{\ctr}{\mathrm{CTR}}
\newcommand{\ctrFE}{{\ctr[F \times E]}}
\newcommand{\ctrRE}{{\ctr[{\cal R}_1 \times E]}}
\newcommand{\ctrRR}{{\ctr[{\cal R}_1 \times {\cal R}_2]}}
\newcommand{\KeyE}{\Key_E}
\newcommand{\KeyF}{\Key_F}
\newcommand{\TimeX}{\mathrm{Time}}
\newcommand{\authh}{{\mathrm{auth}*}}
\newcommand{\LAND}{{\;\wedge\;}}


\newcommand{\gamesfontsize}{\footnotesize}


\newcommand{\soln}[1]{\vspace*{1ex}\fbox{\begin{minipage}[c]{6.0in}{\textbf{\textit{Solution.}}\/#1}\end{minipage}} \vspace*{1ex}}


\newcommand{\mpage}[2]{\begin{minipage}{#1\textwidth}\gamesfontsize #2 \end{minipage}}
\newcommand{\fpage}[2]{\framebox{\begin{minipage}{#1\textwidth}\gamesfontsize #2 \end{minipage}}}

\newcommand{\hfpages}[3]{\hfpagess{#1}{#1}{#2}{#3}}
\newcommand{\hfpagess}[4]{
		\begin{tabular}{c@{\hspace*{.5em}}c}
		\framebox{\begin{minipage}[t]{#1\textwidth}\setstretch{1.1}\gamesfontsize #3 \end{minipage}} 
		&  
		\framebox{\begin{minipage}[t]{#2\textwidth}\setstretch{1.1}\gamesfontsize #4 \end{minipage}}
		\end{tabular}
	}
\newcommand{\hfpagesss}[6]{
		\begin{tabular}{c@{\hspace*{.5em}}c@{\hspace*{.5em}}c}
		\framebox{\begin{minipage}[t]{#1\textwidth}\gamesfontsize #4 \end{minipage}} 
		&  
		\framebox{\begin{minipage}[t]{#2\textwidth}\gamesfontsize #5 \end{minipage}}
		&  
		\framebox{\begin{minipage}[t]{#3\textwidth}\gamesfontsize #6 \end{minipage}}
		\end{tabular}
	}


\def\codestretch{1.1}

\newcommand{\hpagesl}[3]{
	\begin{tabular}{c|c}
	  \begin{minipage}{#1\textwidth}\setstretch{\codestretch} #2 \end{minipage} 
	  & 
	  \begin{minipage}{#1\textwidth} #3 \end{minipage}
	\end{tabular}
	}

\newcommand{\hpagessl}[4]{
	\begin{tabular}{c|c}
	   \begin{minipage}[t]{#1\textwidth}\setstretch{\codestretch} #3 \end{minipage} 
	   & 
	   \begin{minipage}[t]{#2\textwidth}\setstretch{\codestretch} #4 \end{minipage}
	\end{tabular}
	}

\newcommand{\hpages}[3]{
	\begin{tabular}{cc}
	   \begin{minipage}[t]{#1\textwidth}\setstretch{\codestretch} #2 \end{minipage} 
	   & 
	   \begin{minipage}[t]{#1\textwidth}\setstretch{\codestretch} #3 \end{minipage}
	\end{tabular}
	}
\newcommand{\hpagess}[4]{
    \begin{tabular}{cc}
	   \begin{minipage}[t]{#1\textwidth}\setstretch{\codestretch} #3 \end{minipage} 
	   &
	   \begin{minipage}[t]{#2\textwidth}\setstretch{\codestretch} #4 \end{minipage} 
    \end{tabular}
	}

\newcommand{\hpagesss}[6]{
	\begin{tabular}{ccc}
	\begin{minipage}[t]{#1\textwidth}\setstretch{\codestretch} #4 \end{minipage} & 
	\begin{minipage}[t]{#2\textwidth}\setstretch{\codestretch} #5 \end{minipage} &
	\begin{minipage}[t]{#3\textwidth}\setstretch{\codestretch} #6 \end{minipage}
	\end{tabular}}
\newcommand{\hpagesssl}[6]{
	\begin{tabular}{c|c|c}
	\begin{minipage}[t]{#1\textwidth}\setstretch{\codestretch} #4 \end{minipage} & 
	\begin{minipage}[t]{#2\textwidth}\setstretch{\codestretch} #5 \end{minipage} &
	\begin{minipage}[t]{#3\textwidth}\setstretch{\codestretch} #6 \end{minipage}
	\end{tabular}}
\newcommand{\hpagessss}[8]{
	\begin{tabular}{cccc}
	\begin{minipage}[t]{#1\textwidth}\setstretch{\codestretch} #5 \end{minipage} & 
	\begin{minipage}[t]{#2\textwidth}\setstretch{\codestretch} #6 \end{minipage} &
	\begin{minipage}[t]{#3\textwidth}\setstretch{\codestretch} #7 \end{minipage}
	\begin{minipage}[t]{#4\textwidth}\setstretch{\codestretch} #8 \end{minipage}
	\end{tabular}}


\newcommand{\query}[1]{\procfont{query} {#1}:}
\newcommand{\queryl}[1]{\underline{\procfont{query} {#1}:}}
\newcommand{\oracle}[1]{\underline{\procfont{oracle} {#1}:}}
\newcommand{\oraclev}[1]{\underline{\procfont{oracle} {#1}:}\smallskip}
\newcommand{\procedure}[1]{\underline{\procfont{procedure} {#1}:}}
\newcommand{\procedurev}[1]{\underline{\procfont{procedure} {#1}:}\smallskip}
\newcommand{\subroutine}[1]{\underline{\procfont{subroutine} {#1}:}}
\newcommand{\subroutinev}[1]{\underline{\procfont{subroutine} {#1}:}\smallskip}
\newcommand{\subroutinenl}[1]{{\procfont{subroutine} {#1}:}}
\newcommand{\subroutinenlv}[1]{{\procfont{subroutine} {#1}:}\smallskip}
\newcommand{\adversary}[1]{\underline{\procfont{adversary} {#1}:}}
\newcommand{\adversaryv}[1]{\underline{\procfont{adversary} {#1}:}\smallskip}
\newcommand{\experiment}[1]{\underline{{#1}}}
\newcommand{\experimentv}[1]{\underline{{#1}}\smallskip}


\newcommand{\bitsoracle}{{\mathsf{Bits}}}
\newcommand{\indcpa}{{\mathrm{IND}\mbox{-}\mathrm{CPA}}}
\newcommand{\indrcpa}{{\mathrm{IND}\$\mbox{-}\mathrm{CPA}}}
\newcommand{\rorcpa}{\mathrm{RoR}}
\newcommand{\Func}{\mathsf{Func}}

%\newcommand{\Adv}{\mathrm{Adv}}
\newcommand{\ExpUFCMA}[2]{\mathsf{Exp}^{\mathrm{ufcma}}_{#1}{(#2)}}
\newcommand{\ExpRoR}[2]{\mathsf{Exp}^{\mathrm{ror}}_{#1}{(#2)}}
\newcommand{\ExpPRF}[3]{\mathsf{Exp}^{\mathrm{prf}\mbox{-}#3}_{#1}{(#2)}}
\newcommand{\AdvPRF}[2]{\Adv^{\mathrm{prf}}_{#1}(#2)}
\newcommand{\AdvRoR}[2]{\Adv^{\mathrm{ror}}_{#1}(#2)}
\newcommand{\AdvUFCMA}[2]{\Adv^{\mathrm{ufcma}}_{#1}(#2)}

%%%%%%%%%%%%%%%%%%%%%%%%%%%%%%%%%%%%%%%%%%%%%%%%%%%%%%%%%%%%%%%%%%%%%%%%%%%

\title{\bf Problem Set 3 \\[2ex] 
       \rm\normalsize Modern Cryptography, Spring 2013}
\date{}
\author{}
\begin{document}
\maketitle


%%%%%%%%%%%%%%%%%%%%%%%%%%%%%%%%%%%%%%%%%%%%%%%%%%%%%%%%
\vspace*{-1in}

\begin{center}{Group: Binh Tran, David Baldwin\\}
\end{center}

\section*{Problem 1}
Let $E \colon \bits^n \times \bits^n \to \bits^n$ be a blockcipher,
and consider the following three functions, each mapping $\bits^n
\times \bits^n$ to $\bits^n$:

\begin{enumerate}
\item $f_1(V,M) = E_V(V \xor M) \xor M $\\

This function is not collision resistant. Consider the following attack that makes 1 query:\\
\begin{enumerate}
\item $V \gets 0^n$
\item $M_1 \getsr \bits^n$
\item $M_2 \getsr E^-1_V(M_1)$
\item return $(V, M_1)$ and $(V, M_2)$
\end{enumerate}

\textbf{Analysis}

\begin{eqnarray*}
f_1(V, M_1) = E_V(V \xor M_1) \xor M_1\\
=E_V(M_1) \xor M_1
\end{eqnarray*}

\item $f_2(V,M) = E_V(V \xor M) \xor V \xor M$

Consider $f'_2(V,N)=E_V(N) \xor N$ where $N=V \xor M$. $f'_2$ is the same as $f_2$, thus\\If $\exists (V, M), (V', M')$ such that $f_2(V,M)=f_2(V', M')$ then\\ $\exists (V, N), (V', N)$ such that $f'_2(V,N)=f'_2(V', N')$\\

In that case,\\
\begin{eqnarray*}
E_V(N) \xor N &=& E_{V'}(N') \xor N'\\
E_V(N)&=& E_{V'}(N') \xor N' \xor N\\
\end{eqnarray*}

If $E_V$ and $E_{V'}$ are both strong PRF, then 
\begin{eqnarray*}
Pr(N': E_V(N)= E_{V'}(N') \xor N' \xor N)&=&Pr(T: E_V(N)=T)\\
&=&\frac{1}{2^n}
\end{eqnarray*}
and the adversary will need to makes about $\sqrt{2^n}$ queries before the collision happens.
\item $f_3(V,M) = E_V(M) \xor V$
\\This function is also not collision resistant. Consider the following attack that makes 2 queries:\\
\begin{enumerate}
\item $V_1, V_2 \getsr \bits^n$ where $V_1 \neq V_2$\\
\item $M_1 \gets E^-1_{V_1}(V_1)$\\
\item $M_2 \gets E^-1_{V_2}(V_2)$\\
\end{enumerate}

\textbf{Analysis}
\begin{eqnarray*}
E_{V_1}(M_1) \xor V_1 &=& E_{V_1}(E^-1_{V_1}(V_1)) \xor V_1 \\
&=& V_1 \xor V_1 \\
&=& 0^n\\
\end{eqnarray*}
and
\begin{eqnarray*}
E_{V_2}(M_2) \xor V_2 &=& E_{V_2}(E^-1_{V_2}(V_2)) \xor V_2 \\
&=& V_2 \xor V_2 \\
&=& 0^n\\
\end{eqnarray*}
This causes the collision\\
\end{enumerate}

\newcommand{\oPi}{\overline{\Pi}}
\newcommand{\ocalK}{\overline{\mathcal{K}}}
\newcommand{\ocalE}{\overline{\mathcal{E}}}
\newcommand{\ocalD}{\overline{\mathcal{D}}}

\vspace*{-.25in}
\section*{Problem 2} Let~$F \colon \bits^k \times \bits^* \to \bits^n$
be a function family, and let $\Pi = (\calK,\calE,\calD)$ be an
encryption scheme.  Consider the following encryption scheme $\oPi =
(\ocalK,\ocalE,\ocalD)$ where 
\[
\ocalE_{K1,K2}(M) = \calE_{K1}(M)\concat F_{K2}(K1 \concat M)\, .
\]  
To decrypt, $\ocalD_{K1,K2}(C)$
operates as follows: (1) parse~$C$ into $Y \concat T$ where $|T|=n$; (2) $X
\gets \calD_{K1}(Y)$, and if~$X = \bot$ then return~$\bot$; (3) if
$F_{K2}(K1 \concat X)=T$ then return~$X$, else return $\bot$.

\paragraph{Part 1.} Assume that~$F$ is a good PRF, and that $\Pi$ is IND\$-CPA secure.
Are these assumptions sufficient to prove that $\oPi$ is IND\$-CPA
secure?

\newcommand{\INDR}{IND\$-CPA}
\newcommand{\ExpINDz}[2]{\mathsf{Exp}^{\mathrm{ind}\mbox{-}\mathrm{cpa0}}_{#1}{(#2)}}
\newcommand{\ExpINDo}[2]{\mathsf{Exp}^{\mathrm{ind}\mbox{-}\mathrm{cpa1}}_{#1}{(#2)}}
\newcommand{\ExpINDR}[2]{\mathsf{Exp}^{\mathrm{ind\$}\mbox{-}\mathrm{cpa}}_{#1}{(#2)}}
\newcommand{\AdvINDR}[2]{\Adv^{\mathrm{ind\$}\mbox{-}\mathrm{cpa}}_{#1}(#2)}
\newcommand{\Exp}[3]{\mathsf{Exp}^{\mathrm{#1}}_{#2}{(#3)}}
\newcommand{\Adva}[3]{\mathsf{Adv}^{\mathrm{#1}}_{#2}{(#3)}}

\textsc{Answer} 
\\$\oPi$ is not \INDR. Here's an adversary A for $\ExpINDR{\oPi}{(A)}$ that makes 2 query totaling $2n$ bits length, $O(1)$ calculation time:
\begin{enumerate}
\item M $\getsr \bits^n$
\item $X_l \concat X_r \gets X \gets \calO(M)$ where $|X_r| = n$
\item $Y_l \concat Y_r \gets Y \gets \calO(M)$ where $|Y_r| = n$
\item if $X_r = Y_r$ then \texttt{ return 1} else \texttt{ return 0}
\end{enumerate}

\textsc{Analysis}\\
\begin{enumerate}
\item In $\ExpINDo{\oPi}{A}$\\
$X_r = F_{K2}(K1 \concat M)$ and $Y_r = F_{K2}(K1 \concat M)$ by the construction of $\oPi$ thus\\
\begin{center}
 $Pr(\ExpINDo{\oPi}{A}=1) =1$
\end{center}

\item In $\ExpINDz{\oPi}{A}$\\
If $\calO (M)$ is random then A returns 1 when 2 random bit strings has the same suffix 

\begin{eqnarray*}
Pr(\ExpINDo{\oPi}{A}=1) &=& Pr (X_r = Y_r)\\
&=& Pr (X_r = T: Y_r = T)\\
&=& \frac{1}{2^n}
\end{eqnarray*}

\item $\AdvINDR{\oPi}{A}$\\

\begin{eqnarray*}
\AdvINDR{\oPi}{A} &=& Pr(\ExpINDo{\oPi}{A}=1)-Pr(\ExpINDz{\oPi}{A}=1)\\
&=& 1 - \frac{1}{2^n}
\end{eqnarray*}

\end{enumerate}

\paragraph{Part 2.} Are these assumptions sufficient to prove that $\oPi$ is AUTH-secure?
If so, please give a detailed proof sketch (as above), and what you
expect will be the security bound.  If not, please give an efficient attack.

\textsc{Answer}\\
$\oPi$ is AUTH-secure as shown using the following security reduction:

\textit{Statement: } If there exists an adversary A, making $q$ queries totaling $\mu$ bits length and doing calculation in $t$ time, in $\Exp{AUTH}{\oPi}{A}$ then there exists an adversary B, making $q$ queries totaling $/mu + qn$ bits length and doing calculation in $t + O(\frac{\mu}{q})$ time, in $\Exp{PRF}{F}{B}$ that satisfy

\begin{eqnarray*}
\Adva{PRF}{F}{B} = \Adva{AUTH}{\oPi}{A} -  \frac{1}{2^n}
\end{eqnarray*}

\textit{Building B:}\\
\begin{itemize}
\item $K_1 \gets \bits^n$
\item Run A
\item If A asks for encryption of M
\begin{itemize}
\item $X \gets \calE_{K_1}(M)$ where $\calE$ is simulated by B
\item $Y \gets \calO(K_1 \concat M)$
\item give $(X \concat Y)$ to A
\end{itemize}
\item If A asks for decryption of Z
\begin{itemize}
\item $C \concat D \gets Z$ where $|D|=n$
\item $I \gets \calE^{-1}_{K_1}(C)$ where $\calE^{-1}$ is simulated by B
\item $J \gets \calO(K_1 \concat I)$
\item if $D==J$ then gives $I$ to A else give $\bot$ to A
\end{itemize}
\item Run until A halts.
\item $C_0 \concat T_0 \gets A$ where $|T_0|=n$
\item $M_0 \gets \calE^{-1}_{K_1}(C_0)$ where $\calE^{-1}$ is simulated by B
\item $T_1 \gets \calO(M_0)$
\item if $T_0 == T_1$ returns 1 else return 0
\end{itemize}

\textit{Analysis}\\
\begin{itemize}
\item In $\Exp{PRF-1}{F}{B}$\\
Since B simulate $\oPi$ exactly, and B only returns 1 when A manages to break $\oPi$ in $Exp{AUTH}{\oPi}{A}$
\begin{equation}\label{prf1}
Pr(\Exp{PRF-1}{F}{B}=1)= \Adva{AUTH}{\oPi}{A}
\end{equation}
\item In $\Exp{PRF-0}{F}{B}$\\
In this case, B returns 1 when a tag chosen randomly happens to match the tag returned by A\\
\begin{equation}\label{prf0}
Pr(\Exp{PRF-0}{F}{B}=1) = \frac{1}{2^n}
\end{equation}
\item $\Adva{PRF}{F}{B}$\\
From \ref{prf1} and \ref{prf0}
\begin{equation}
\Adva{PRF}{F}{B} = \Adva{AUTH}{\oPi}{A} -  \frac{1}{2^n}
\end{equation}
\item Run time behavior:\\
\begin{itemize}
\item Every time A asks a query, either to the Encrypt or Decrypt oracle, B asks $\calO$ a query. Thus B makes $q$ queries in total.
\item Every time A asks a $m$-bit $M$ query, B makes a $(|K \concat M| = n + m)$-bit query. After $q$ queries, the total length is $\mu + qn$ bits
\item Every time A asks a $l$-block query, B has to spend an extra $O(l)$ time to encrypt / decrypt those blocks. In the end A queries total $\frac{\mu}{q}$ blocks thus B spends $t+O(\frac{\mu}{q})$ time
\end{itemize}
\end{itemize}

\paragraph{Part 3.}
$\ocalE'_{K1,K2}(M) = \calE_{K1}(M)\concat F_{K2}(K1 \concat |M|)$ is secure in neither IND\$-CPA nor AUTH\\

For IND\$-CPA, the same attack on $\oPi$ also works on $\oPi'$ The reason is that the tag in $\oPi'$ is generated by the length of the plain text, thus same plain text - having the same length - will result in same tag.\\

The following attacker can bypass $\oPi'$ in AUTH game - Assume that $|X|=|Y| \rightarrow |\calE(X)|=|\calE(Y)|$

\begin{itemize}
\item $M_1 \getsr \bits^n$
\item $C_1 \concat T_1 \gets \calE'(M_1)$ where $|T_1|=n$
\item $C_2 \getsr \bits^{n}$
\item return $C_2 \concat T_1$
\end{itemize}

The reason this works is that $T_1$ is the universal tag for all n-bit plain text, and by our assumption, n-bit cipher text. $C_2$ is one of such cipher text, and thus will be accepted by the decryption.

\end{document}
